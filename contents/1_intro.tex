\chapter{Introduzione}
\begin{comment}
con il problema e l'idea non ci siamp proprio..
la scaletta dei capitoli è:
- introduzione
- stato dell'arte
- metodi e strumenti
- sviluppo del progetto
- risultati

Sottosezioni di introduzione:
- Motivaione

    - modelli neurali per la segmentazione
        - applicazioni
        - il dataset e i requisiti per l'addestramento

- Approccio al problema

    - La generazione di un dataset sintetico

- introduzione alle reti neurali

    - feedforward networks
        - il neurone
        - la funzione di attivazione
        - backpropagation

    - convolutional neural networks
        - la convoluzione
        - il pooling
        - l'upsampling

\end{comment}


\section{Il problema \ok}
\begin{comment}

\end{comment}
Oggi nello scenario industriale e di ricerca, per l'addestramento di modelli neurali una delle 
principali difficoltà è la mancanza di dataset adatti al proprio task, i quali per la segmentazione di immagini possono
essere particlarmente difficili da reperire o realizzare, in quanto richiedono un elevato numero di immagini annotate.
Più il task è complesso più il dataset necessario per addestrare un modello sarà grande,
e potenzialmente rendendo ogni singolo esempio più difficile da realizzare, richiedendo una quantità di ore uomo molto elevata.

Uno dei task in cui oggi i modelli neurali trovano un vasto impiego è la segmentazione dei difetti, principalmente in campo industriale,
i quali sono utilizzati per verificare automaticamente la qualità di un prodotto o di un semilavorato, per ridurre i costi e i tempi del controllo manuale,
il quale è spesso prono ad errori in quanto l'uomo è soggetto a stanchezza o a mancanza di concentrazione,
al contrario un modello con un'elevata accuratezza è in grado di riconoscere un difetto con prestazioni costanti.

La problematica di avere un modello con elevata accuratezza per task di questo tipo è relativa alla preparazione dei dati.
Molti task richiedono molta concentrazione da parte dell'annotatore in quanto non sempre i difetti sono ben visibili e dunque ogni singolo
esempio deve essere controllato attentamente anche da più persone, e questo rende il processo di annotazione molto lungo e costoso.
Spesso in oltre anche dopo diversi controlli alcuni difetti possono comunque sfuggire al processo di annotazione.


\section{L'idea \ok}
\begin{comment}
Realizzare un architettura in grado di generare un buona quantità di dati annotati a partire da una quantità ridotta.
TODO: linka paper google su "copy paste technique"
\end{comment}
L'obbiettivo di questo progetto è proporre una soluzione a questo problema, o almeno mitigare la sua gravità.
Per far ciò si propone di realizzare un architettura in grado di generare un oggetto in un'immagine presistente, specificando la maschera di quest'ultimo.

Tale architettura deve essere in grado di prendere in input un'immagine e una maschera, e generare l'oggetto all'interno della maschera, 
lasciando il più possibile invariata l'immagine al di fuori dell'area d'interesse.
Il modello dovrebbe essere in grado di generare un oggetto all'interno dell'area obbiettivo seguendo la distribuzione morfologica degli oggetti 
presenti nel dataset di addestramento.

Tale tecnica è principalmente pensata per tutti quei casi in cui non è semplice o possibile realizzare sinteticamente un oggetto a partire da metodi deterministici, 
ad esempio per la segmentazione dei difetti, i quali non possono essere copiati e incollati in una nuova mmagine senza che sia presente un evidente linea di separazione
tra il difetto e il resto dell'immagine, e quindi non possono essere generati con una tecnica di tipo copy paste, senza introdurre artefatti, che potrebbero 
portare ad un peggioramento delle prestazioni del modello, in quanto questo associerebbe il difetto alla linea di separazione.

In questo progetto il dataset utilizzato è Severstal steel defect detection, il quale presenta campioni di acciaio più tosto simili,
ma in generale questa tecnica potrebbe essere utilizzata anche per casi più complessi come il rilevmento di danni su veicoli o edifici,
dove la morfologia del difetto è comune, ma la morfologia, il colore e la texture dell'oggetto su cui deve essere applicato sono differenti 
dal dataset di addestramento, in quest'ultimo scenario infatti non è possibile utilizzare una tecnica di tipo copy paste, in quanto il difetto 
non potrebbe essere applicato in modo coerente con l'immagine base.