\chapter{Dataset e architetture base utilizzate\ok \ok \ok}

\section{Il Dataset: Severstal steel defect detection \ok}
\begin{comment}
dataset utilizzato per il progetto
\end{comment}
L'acciaio è uno dei materiali più comunemente utilizzati in tutto il mondo, e la sua produzione è in continua crescita.
La sua versatilità e la sua resistenza lo rendono un materiale molto utilizzato in diversi settori, come l'edilizia, l'automotive, l'industria elettronica, l'industria aerospaziale, ecc.
Per produzioni su larga scala di acciaio come di altri materiali o prodotti, è necessario che il materiale sia di qualità, e che non contenga difetti,
ma è difficile per gli operatori umani rilevare difetti come graffi, crepe, ecc. con elevata affidabiltà, per tale ragione 
è necessario adottare sistemi automatizzati che siano in grado di rilevare difetti in modo affidabile e veloce.
Severstal è una delle principali aziende produttrici di acciaio in Russia, e produce circa 10 milioni di tonnellate di acciaio all'anno,
quest'ultima ha iniziato dunque ad utilizzare il machine learning per automatizzare il processo di rilevazione dei difetti, e ha messo a disposizione
questo dataset nel 2019 per permettere a chiunque di partecipare alla challenge, e di testare le proprie idee, sperando di riuscire ad aumentare l'affidabiltà
del proprio sistema di rilevazione difetti.

Questo dataset è diviso in 2 parti, training e test set, ma per il test set non sono stati rilasciati i ground truth, quindi non è possibile testare
il modello finale sul test set originale e verranno dunque utilizzati solo i dati del training set, i quali verranno suddivisi in training e test set.
Il training set originale  è composto da 12568 immagini, che verrano divise in 50\% per il nuovo training set e 50\% per il nuovo test set, 
quindi un totale di 6284 immagini per il training set e 6284 immagini per il test set.

il training set verrà utilizzato per il training del generatore di difetti sintetici, mentre il test set verrà utilizzato per testare 
il modello di segmentazione finale addestrato con il dataset sintetico.


\section{Lama:  \ok}
\begin{comment}
Architettura base utilizzata per il modello.
\end{comment}


\section{Stylegan2:  \ok}
\begin{comment}
Architettura usata per generare le maskere sintetiche
e come componente del modello finale.
\end{comment}

