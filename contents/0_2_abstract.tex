\clearpage
\phantom{a}
\vfill


\chapter{Prefazione \ok \ok \ok}


\begin{flushleft}


\begin{comment}



\end{comment}


Il mio percorso nel campo dell'inteligenza artificiale è iniziato diversi anni fà, alle superiori per l'esattezza,
dove sentii per la prima volta parlare di reti neurali, ad un corso pomeridiano voluto dal prof. Roberto Lulli il quale mi ha mostrato per primo questo affascinante campo di ricerca.

Ho svolto durante il mio percorso di studi diversi progetti incentrati su questa tematica, partendo da semplici reti neurali,
e confrontandomi con progetti sempre più complessi fino ad arrivare ai modelli generativi basati sull'architettura GAN (Generative Adversarial Network),
del quale in questa tesi proporò una variante.

Lo scopo di questa tesi è quello di proporre un rimedio ad uno dei grandi problemi che affligge le aziende che si occupano di addestrare modelli neurali per la segmentazione di immagini,
ovvero la difficoltà nel reperire immagini annotate ed gli elevatissimi costi per la realizzazione di un dataset ad hoc per il proprio task, problema con il quale mi sono dovuto confrontare io stesso nell'ultimo
anno come sviluppatore presso l'azienda Cloe.ai.


\end{flushleft}



\vfill
\newpage