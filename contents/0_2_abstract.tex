\clearpage
\phantom{a}
\vfill


\chapter{Prefazione \ok \ok \ok}


\begin{flushleft}


\begin{comment}



\end{comment}


Il mio percorso nel campo dell'inteligenza artificiale è iniziato diversi anni fà, alle superiori per l'esattezza,
dove sentii per la prima volta parlare di reti neurali, ad un corso pomeridiano voluto dal prof. Roberto Lulli il quale mi ha mostrato per primo questo affascinante campo di ricerca.

Ho svolto durante il mio percorso di studi diversi progetti incentrati su questa tematica, partendo da semplici reti neurali,
e confrontandomi con progetti sempre più complessi fino ad arrivare ai modelli generativi basati sull'architettura GAN (Generative Adversarial Network),
del quale in questa tesi proporò una variante.

Lo scopo di questa tesi è quello di investigare la fattibilità di una potenziale soluzione ad uno dei grandi problemi che affligge oggi le aziende che si occupano di addestrare modelli neurali per la segmentazione di immagini,
ovvero la difficoltà nel reperire immagini annotate, le quali hanno elevatissimi costi di realizzazione.

Tale scelta è stata naturale, in quanto ho dovuto confrontarmi in prima persona con questo problema nell'ultimo anno, come sviluppatore presso l'azienda Cloe.ai.
In questa esperienza ho gestito per quasi un anno la realizzazione di un complesso dataset per l'addestramento di un modello di segmentazione di difetti, 
tale dataset aveva dei requisiti molto alti, e al contempo le risorse per realizzarlo erano limitate, per tale ragione ho potuto comprendere affondo le problematiche legate a questo tipo di progetto, 
facendo i conti io stesso con le spese e i progressi ottenuti.

La realizzazione di un dataset su larga scala è un'operazione molto complessa, che richiede una elevata coordinazione tra annotatori, revisori, sviluppatori,
 e un'accurata documentazione che in base al problema può richiedere anche diversi mesi per poter essere redatta efficaciemente.
Tutto cio mi ha fornito la motivazione per cercare una soluzione per accorciare questo lungo e tedioso processo e dunque attenuare
gli ingenti costi che un'azienda deve sostenere per realizzare un dataset di questo tipo. 

\end{flushleft}



\vfill
\newpage