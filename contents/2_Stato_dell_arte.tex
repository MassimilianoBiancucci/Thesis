\chapter{Stato dell'arte}
\begin{comment}
Cosa devo mettere in questa sezione??

- L'obbiettivo è arrivare a parlare del modello LAMA e stylegan2, quindi effetturò una serie di passaggi per passare dai primi modelli gan generativi
(goodfellow2014generative) a quelli più recenti come stylegan2 (karras2019analyzing), che erano i più utilizzati per generare immagini di alta qualità fino a poco tempo fa.
Per poi passare per pix2pix per introdurre le reti generative condizionate, e quindi dopo una breve introduzione al problema dell'inpainting arrivare a parlare di LAMA.
\end{comment}

In questa sezione verranno mostrati i principali modelli generativi basati su architettura GAN, che sono stati utilizzati negli ultimi anni, partendo dal 
primo modello proposto da Goodfellow et al. \cite{goodfellow2014generative} capostipite di questa famiglia di modelli passando per le principali innovazioni
proposte da altri autori, fino ad arrivare a modelli più recenti come StyleGAN2 \cite{karras2020analyzing} e LAMA \cite{suvorov2021resolutionrobust}.

\section{Il primo modello gan}
Nel 2014 Ian Goodfellow et al. \cite{goodfellow2014generative} hanno proposto il primo modello generativo basato su architettura GAN (Generative Adversarial Network).),
Questa pubblicazione ha segnato un punto di svolta nella ricerca dei modelli di deep learning generativi, i quali fino ad allora erano
basati su architetture ad autoencoder. La differenza sostanziale tra i due tipi di modelli è che i modelli basati su autoencoder 
